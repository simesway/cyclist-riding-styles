\chapter{Methodology}

The pipeline consists of three stages: maneuver extraction, local riding style clustering, maneuver level clustering and conditioned analysis. Each module was designed to ensure reproducibility, robustness and flexibility.

\section{Dataset}
Dataset Description \& Data preprocessing

\section{Interaction Retrieval}
The Analysis includes the extraction of special maneuvers (following, overtaking), but also the calculation of safety measures between cyclists and other traffic participants (cyclists, cars, pedestrians). Therefore encounters between traffic participants have to be extracted from the trajectory dataset. An encounter is a pair of traffic participants that coexist in a single timestamp \cite{perez_castro_empirical_2025} within a certain distance \cite{mohammed_characterization_2019}.

To efficiently index and search for potential trajectory encounters, a Spatio-Temporal R-Tree (STR-Tree) was used \cite{deng_trajectory_2011}, which serves as an optimized spatial indexing structure for querying spatial relationships between trajectories. 

Trajectories were initially simplified using the Douglas-Peucker algorithm, which reduces the number of trajectory points while maintaining the characteristic shape of the movement  \cite{zheng_trajectory_2015}. The simplified trajectories are then split into segments at the remaining critical points, effectively reducing computational overhead while preserving the fundamental geometric properties of the original trajectories.

All the resulting segments were then used to construct an STR-Tree, which enables efficient spatial intersection queries. Given a target trajectory, a buffer zone is defined around it with a user-specific distance. The STR-Tree is queried for all segments intersecting this buffer, yielding candidate objects that may interact with the target.
To manage computational resources, the analysis was constrained to small batches of target trajectories, with the time window limited to maximum time window of the target trajectories.

For each candidate, relative distances and speeds are computed at matching timestamps. These encounter metrics enable further domain specific filtering on these interaction candidates. The structured output can be used for further analysis, such as interaction and maneuver identification and the computation of surrogate safety metrics.


\section{Maneuvers}

All interaction candidates were additionally filtered by the three selection criteria from Kutsch et al. \cite{kutsch_analyzing_2025}:
\begin{enumerate}
	\item Distance Constraint: Identified encounters with at least one timestamp where the absolute distance between center points was less than 20 m, while ensuring all timestamps maintained a maximum distance of 40 m.
	\item Temporal Criterion: Minimum 5-second trajectory overlap.
	\item Movement Validation: Each trajectory moves at least 10m.
	\item Heading Criterion: was used inside maneuver detection code
\end{enumerate} 

Additionally a constraint was set to exclude non-moving objects, by only keeping timestamps where the velocity of an object is above the average human walking speed of 1.4-1.48 m/s² \cite{bartels_bewegungsverhalten_2014}.

For a pair of encountering vehicles, trajectories were synchronized by timestamp and the lateral and longitudinal distances are computed relative to a chosen vehicle's frame of reference:
$$formula?$$
Most papers characterize a Following and Overtaking Maneuver by the longitudinal and lateral distances from the perspective of the initially following cyclist \cite{khan_characteristics_2001, kutsch_analyzing_2025, mocsari_analysis_2009, mohammed_characterization_2019}. This transformation ensures that subsequent analysis is invariant to global orientation and focuses on relative motion. Therefore map matching is not crucial for this part of the analysis and can work standalone.
By smoothing the longitudinal and lateral distances using a Gaussian temporal kernel of 0.6s yielded the best results for determining the start and end points of a maneuver.

\subsection{Following}

The longitudinal displacement signal is analyzed for sign flips, which indicate transitions in relative positioning \cite{hoogendoorn_bicycle_2016, mohammed_characterization_2019}. The trajectory is divided into contiguous intervals between these zero crossings to focus on periods of consistent follower-leader ordering.

Additional Spatial Constraints: Lateral distance always below 1m, Longitudinal distance between 1 and 30 meters.
Temporal Constraints: Time Headway (THW) of a maximum of 6 seconds \cite{kutsch_analyzing_2025} and a minimum duration of 1s.
A Directional Constraint of the relative heading being under 35 degrees in difference \cite{feng_automatic_2023, kutsch_analyzing_2025}.
The candidates fulfilling all criteria for 13 consecutive timestamps (1 second) are labeled as Following Maneuvers.

\subsection{Overtaking}
Longitudinal signals are analyzed to detect sign flips from positive to negative, which indicate potential overtake events \cite{kutsch_analyzing_2025, mohammed_characterization_2019}. To improve robustness, spurious flips are filtered by requiring sign persistence over multiple frames. Nearby events are merged to handle fluctuations around 0.
Each candidate crossing defines a window of interest, centered on the longitudinal sign flip. The candidates window is refined by:

\begin{compactitem}
	\item Selecting start and end points based on local maxima of the longitudinal distance.
	\item Clamping window size to a maximum frame limit to ensure temporal consistency.
\end{compactitem}

The algorithm to find maneuver edges proposed by Kutsch et al. \cite{kutsch_analyzing_2025} was applied with various thresholds to these windows of interest, but it was found that this algorithm is very sensitive for noise and thus difficult to fine tune in a pipeline that should be mostly automatic. 

Therefore a own approach was found, that is supposed to be more robust to noise in datasets, but also at the cost of non optimal maneuver-edge detection. This approach was preferred because of its simplicity and making the edge detection more robust on varying datasets, since fine tuning the hyperparameters is not necessary. Kutsch et al. also manually checked and verified the candidates for each hyperparameter, a process that is very costly and not possible within an automatic pipeline.

The goal of the process is to find the last point where the overtaker is laterally behind the overtaken cyclist, with the start (or end) being the point where the lateral distance is closest to 0:

The absolute lateral displacement signal is used to determine edges of the maneuver, applying a logarithmic set of thresholds. This identifies when the overtaking vehicle departs from and returns to its lane or lateral baseline. The Start (or End) is determined by the timestamp where the abs lateral distance last is below the smallest threshold (first is below the smallest threshold) [see figure whatever]. The Thresholds are scaled logarithmically between the lateral distance at overtaking point and 0. The optimal number of thresholds was found to be 7 with a highest threshold of 1.5 m.

Candidate maneuvers are validated using:
\begin{compactitem}
	\item Monotonicity checks on longitudinal displacement.
	\item Distance constraints on lateral offset (<3m and >1m) \cite{perez_castro_empirical_2025}.
	\item Distance between 0.3 m and 3 m at overtaking point
	\item Duration thresholds to remove extremely short events.
\end{compactitem}


For validated candidates, the module extracts descriptive statistics \cite{khan_characteristics_2001}:
\begin{compactitem}
	\item Temporal features: start, crossing, and end timestamps; duration.
	\item Spatial features: minimum, maximum longitudinal distances; lateral offsets at key points.
	\item Kinematic features: mean speeds of follower and leader, relative speed, maximum follower acceleration.
\end{compactitem}

\section{Hierarchical Riding Style Identification}

For the following stages of the analysis, the maneuvers were seperated. First the maneuvers were segmented using the sliding window approach, resulting in fixed-length temporal windows of 1s duration with 50\% overlap \cite{li_volatility-based_2025, mohammadnazar_classifying_2021}, a common approach for modeling short-term driving and riding dynamics while preserving temporal continuity. For each window three sets of features are extracted and aggregated, each used by a different stage of the analysis and separated cleanly to ensure domain separation. The feature sets are Riding/Volatility, Traffic and Infrastructure descriptive features. 

\subsection*{Level 1: Local Riding Behaviour (Window Level)}
\label{meth:level1}

Within each window, magnitude- \cite{enev_automobile_2016} and volatility-based features are extracted from kinematic signals.

Magnitude features describe the intensity of motion and include speed, longitudinal acceleration, and yaw rate. For each of these signals, minimum and maximum values within the window are computed.

Volatility features capture short-term variability and dynamic fluctuations. They are derived from speed, longitudinal and lateral acceleration, and rotational fluctuations. For each signal, the mean, mean absolute deviation, coefficient of variation, and quartile coefficient of variation are calculated.

All features are standardized, and magnitude and volatility features are weighted equally. Feature relevance is assessed using feature selection techniques and principal component analysis. The resulting window-level feature vectors are clustered separately per maneuver type. 

Yarlagadda and Pawar \cite{yarlagadda_exploring_2022} performed a two level clustering analysis seperating magnitude and volatility features.  In this paper magnitude and volatility features are clustered together, which allows the model to learn characteristic couplings between riding intensity and variability, yielding a set of local riding regimes that are globally consistent.

The extracted feature vectors are standardized and optionally reduced in dimensionality using principal component analysis (PCA). Two strategies are supported: (i) joint PCA across all features or (ii) separate PCAs for magnitude and volatility features, allowing variance to be captured individually per feature type. Cluster assignments are computed using K-means, with the optimal number of clusters determined automatically via a combination of the elbow method and silhouette scores. Each sliding window is then assigned a local riding regime corresponding to its cluster, producing a globally consistent segmentation of riding behavior. Cluster centers can be interpreted in the original feature units, facilitating analysis of characteristic riding patterns. Feature indices for magnitude and volatility are automatically managed to support both split and joint analyses.

Lower-dimensional PCA (60–80\% variance) produces more robust and interpretable clusters. Increasing dimensionality beyond 80–95\% adds minor variation but reduces separation quality and increases overlap, without providing additional behavioral insight.

\subsection*{Level 2: Maneuver-Level Representation}

At the maneuver level, each maneuver is represented as an aggregation of the previously identified local regimes. Specifically, the distribution of regimes within a maneuver is computed as the proportion of time spent in each regime. In addition, transition frequencies between regimes are extracted to capture dynamic changes in riding behaviour over the course of a maneuver.

Maneuvers were seperated in two modes: active and passive, where passive maneuvers are maneuvers that experienced only a single riding regime. Whereas active maneuvers experience at least two riding regimes and can therefore be analyzed further.

Following the state-based view of human behavior proposed by Pentland and Liu \cite{pentland_modeling_1999}, riding behavior is represented as transitions between a finite set of local regimes, where maneuvers emerge from the temporal composition of these regimes rather than from raw kinematic signals alone.

Therefore only active maneuvers were further clustered to differentiate between modes in active overtaking manuevers, since including passive maneuvers won't yield much insight in transition behaviour of local riding regimes.

These regime-based descriptors are complemented by maneuver-level features such as mean and 90th percentile speed, minimum, maximum, and standard deviation of lateral displacement, mean time headway, relative speed, and further context-dependent variables. Together, these features provide a compact yet expressive summary of how a maneuver is executed.

\subsection*{Level 3: Infrastructure-Conditioned Analysis}

In this stage the goal is to find correlation between local regimes and traffic, infrastructure features.

Features describing the infrastructure and environment are incorporated at the window level \ref{meth:level1} in the feature vector. Each window inherits an infrastructure label corresponding to the road or path segment on which it occurs, while the regime definitions learned at Level 1 remain fixed. This makes the analysis of local riding behavior

For each maneuver, regime statistics are then computed separately for each infrastructure type, such as the proportion of time spent in each regime or regime transition patterns conditioned on infrastructure. This design enables a direct comparison of riding behavior across different infrastructures while preserving a common behavioral reference frame, which is essential for isolating infrastructure-induced adaptations from intrinsic riding style.
features:
distance to traffic light (not intersection because directly extractable from xodr) within 60 meters (first fixation of traffic light between 35 and 60m \cite{rupi_visual_2019})
collision avoidance and lane deviation \cite{kuderer_learning_2015}
traffic flow state key factor influencing style \cite{ren_lane-change_2025}
